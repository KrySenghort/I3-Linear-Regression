% Options for packages loaded elsewhere
\PassOptionsToPackage{unicode}{hyperref}
\PassOptionsToPackage{hyphens}{url}
%
\documentclass[
]{article}
\usepackage{amsmath,amssymb}
\usepackage{lmodern}
\usepackage{iftex}
\ifPDFTeX
  \usepackage[T1]{fontenc}
  \usepackage[utf8]{inputenc}
  \usepackage{textcomp} % provide euro and other symbols
\else % if luatex or xetex
  \usepackage{unicode-math}
  \defaultfontfeatures{Scale=MatchLowercase}
  \defaultfontfeatures[\rmfamily]{Ligatures=TeX,Scale=1}
\fi
% Use upquote if available, for straight quotes in verbatim environments
\IfFileExists{upquote.sty}{\usepackage{upquote}}{}
\IfFileExists{microtype.sty}{% use microtype if available
  \usepackage[]{microtype}
  \UseMicrotypeSet[protrusion]{basicmath} % disable protrusion for tt fonts
}{}
\makeatletter
\@ifundefined{KOMAClassName}{% if non-KOMA class
  \IfFileExists{parskip.sty}{%
    \usepackage{parskip}
  }{% else
    \setlength{\parindent}{0pt}
    \setlength{\parskip}{6pt plus 2pt minus 1pt}}
}{% if KOMA class
  \KOMAoptions{parskip=half}}
\makeatother
\usepackage{xcolor}
\usepackage[margin=1in]{geometry}
\usepackage{color}
\usepackage{fancyvrb}
\newcommand{\VerbBar}{|}
\newcommand{\VERB}{\Verb[commandchars=\\\{\}]}
\DefineVerbatimEnvironment{Highlighting}{Verbatim}{commandchars=\\\{\}}
% Add ',fontsize=\small' for more characters per line
\usepackage{framed}
\definecolor{shadecolor}{RGB}{248,248,248}
\newenvironment{Shaded}{\begin{snugshade}}{\end{snugshade}}
\newcommand{\AlertTok}[1]{\textcolor[rgb]{0.94,0.16,0.16}{#1}}
\newcommand{\AnnotationTok}[1]{\textcolor[rgb]{0.56,0.35,0.01}{\textbf{\textit{#1}}}}
\newcommand{\AttributeTok}[1]{\textcolor[rgb]{0.77,0.63,0.00}{#1}}
\newcommand{\BaseNTok}[1]{\textcolor[rgb]{0.00,0.00,0.81}{#1}}
\newcommand{\BuiltInTok}[1]{#1}
\newcommand{\CharTok}[1]{\textcolor[rgb]{0.31,0.60,0.02}{#1}}
\newcommand{\CommentTok}[1]{\textcolor[rgb]{0.56,0.35,0.01}{\textit{#1}}}
\newcommand{\CommentVarTok}[1]{\textcolor[rgb]{0.56,0.35,0.01}{\textbf{\textit{#1}}}}
\newcommand{\ConstantTok}[1]{\textcolor[rgb]{0.00,0.00,0.00}{#1}}
\newcommand{\ControlFlowTok}[1]{\textcolor[rgb]{0.13,0.29,0.53}{\textbf{#1}}}
\newcommand{\DataTypeTok}[1]{\textcolor[rgb]{0.13,0.29,0.53}{#1}}
\newcommand{\DecValTok}[1]{\textcolor[rgb]{0.00,0.00,0.81}{#1}}
\newcommand{\DocumentationTok}[1]{\textcolor[rgb]{0.56,0.35,0.01}{\textbf{\textit{#1}}}}
\newcommand{\ErrorTok}[1]{\textcolor[rgb]{0.64,0.00,0.00}{\textbf{#1}}}
\newcommand{\ExtensionTok}[1]{#1}
\newcommand{\FloatTok}[1]{\textcolor[rgb]{0.00,0.00,0.81}{#1}}
\newcommand{\FunctionTok}[1]{\textcolor[rgb]{0.00,0.00,0.00}{#1}}
\newcommand{\ImportTok}[1]{#1}
\newcommand{\InformationTok}[1]{\textcolor[rgb]{0.56,0.35,0.01}{\textbf{\textit{#1}}}}
\newcommand{\KeywordTok}[1]{\textcolor[rgb]{0.13,0.29,0.53}{\textbf{#1}}}
\newcommand{\NormalTok}[1]{#1}
\newcommand{\OperatorTok}[1]{\textcolor[rgb]{0.81,0.36,0.00}{\textbf{#1}}}
\newcommand{\OtherTok}[1]{\textcolor[rgb]{0.56,0.35,0.01}{#1}}
\newcommand{\PreprocessorTok}[1]{\textcolor[rgb]{0.56,0.35,0.01}{\textit{#1}}}
\newcommand{\RegionMarkerTok}[1]{#1}
\newcommand{\SpecialCharTok}[1]{\textcolor[rgb]{0.00,0.00,0.00}{#1}}
\newcommand{\SpecialStringTok}[1]{\textcolor[rgb]{0.31,0.60,0.02}{#1}}
\newcommand{\StringTok}[1]{\textcolor[rgb]{0.31,0.60,0.02}{#1}}
\newcommand{\VariableTok}[1]{\textcolor[rgb]{0.00,0.00,0.00}{#1}}
\newcommand{\VerbatimStringTok}[1]{\textcolor[rgb]{0.31,0.60,0.02}{#1}}
\newcommand{\WarningTok}[1]{\textcolor[rgb]{0.56,0.35,0.01}{\textbf{\textit{#1}}}}
\usepackage{graphicx}
\makeatletter
\def\maxwidth{\ifdim\Gin@nat@width>\linewidth\linewidth\else\Gin@nat@width\fi}
\def\maxheight{\ifdim\Gin@nat@height>\textheight\textheight\else\Gin@nat@height\fi}
\makeatother
% Scale images if necessary, so that they will not overflow the page
% margins by default, and it is still possible to overwrite the defaults
% using explicit options in \includegraphics[width, height, ...]{}
\setkeys{Gin}{width=\maxwidth,height=\maxheight,keepaspectratio}
% Set default figure placement to htbp
\makeatletter
\def\fps@figure{htbp}
\makeatother
\setlength{\emergencystretch}{3em} % prevent overfull lines
\providecommand{\tightlist}{%
  \setlength{\itemsep}{0pt}\setlength{\parskip}{0pt}}
\setcounter{secnumdepth}{-\maxdimen} % remove section numbering
\usepackage{fancyhdr}
\pagestyle{fancy}
\fancyhead[LO]{Institute of Technology of Cambodia}
\fancyhead[RO]{Regression Analysis}
\fancyfoot[CO]{}
\fancyfoot[LO]{Dr. Phauk Sokkhey and Mr. Nhim Malai}
\fancyfoot[RO]{\thepage}
\usepackage{float}

\ifLuaTeX
  \usepackage{selnolig}  % disable illegal ligatures
\fi
\IfFileExists{bookmark.sty}{\usepackage{bookmark}}{\usepackage{hyperref}}
\IfFileExists{xurl.sty}{\usepackage{xurl}}{} % add URL line breaks if available
\urlstyle{same} % disable monospaced font for URLs
\hypersetup{
  hidelinks,
  pdfcreator={LaTeX via pandoc}}

\author{}
\date{\vspace{-2.5em}}

\begin{document}

\begin{center}
\textbf{I3-TD2} \\
\textbf{Simple Linear Regression}
\end{center}

\hypertarget{problem-1}{%
\section{Problem 1}\label{problem-1}}

\textbf{Height and weight data} (Data file: \texttt{Htwt}) The table
below and the the data file give \texttt{ht} = height in centimeters and
\texttt{wt} = weight in kilograms for a sample of \(n=10\) 18-year-old
girls. The interest is in predicting weight from height.

\begin{Shaded}
\begin{Highlighting}[]
\FunctionTok{library}\NormalTok{(alr4)}
\FunctionTok{data}\NormalTok{(Htwt)}

\NormalTok{Htwt}
\end{Highlighting}
\end{Shaded}

\begin{verbatim}
##       ht   wt
## 1  169.6 71.2
## 2  166.8 58.2
## 3  157.1 56.0
## 4  181.1 64.5
## 5  158.4 53.0
## 6  165.6 52.4
## 7  166.7 56.8
## 8  156.5 49.2
## 9  168.1 55.6
## 10 165.3 77.8
\end{verbatim}

\begin{enumerate}
\def\labelenumi{\alph{enumi}.}
\tightlist
\item
  Draw a scatterplot of \texttt{wt} on the vertical axis versus
  \texttt{ht} on the horizontal axis. On the basis of this plot, does a
  simple linear model make sense for these data? Why or why not?
\end{enumerate}

\begin{Shaded}
\begin{Highlighting}[]
\FunctionTok{plot}\NormalTok{(Htwt}\SpecialCharTok{$}\NormalTok{wt}\SpecialCharTok{\textasciitilde{}}\NormalTok{Htwt}\SpecialCharTok{$}\NormalTok{ht)}
\end{Highlighting}
\end{Shaded}

\includegraphics{TD2RAA_files/figure-latex/unnamed-chunk-2-1.pdf}

\begin{enumerate}
\def\labelenumi{\alph{enumi}.}
\setcounter{enumi}{1}
\tightlist
\item
  Compute estimates of the slope and the intercept for the regression of
  \(Y\) on \(X\). Draw the fitted line on your scatterplot.
\end{enumerate}

\begin{Shaded}
\begin{Highlighting}[]
\NormalTok{lm\_Htwt}\OtherTok{\textless{}{-}}\FunctionTok{lm}\NormalTok{(wt}\SpecialCharTok{\textasciitilde{}}\NormalTok{ht, }\AttributeTok{data=}\NormalTok{Htwt)}
\NormalTok{lm\_Htwt}
\end{Highlighting}
\end{Shaded}

\begin{verbatim}
## 
## Call:
## lm(formula = wt ~ ht, data = Htwt)
## 
## Coefficients:
## (Intercept)           ht  
##    -36.8759       0.5821
\end{verbatim}

\begin{Shaded}
\begin{Highlighting}[]
\FunctionTok{plot}\NormalTok{(Htwt}\SpecialCharTok{$}\NormalTok{wt}\SpecialCharTok{\textasciitilde{}}\NormalTok{Htwt}\SpecialCharTok{$}\NormalTok{ht)}
\FunctionTok{abline}\NormalTok{(}\FunctionTok{lm}\NormalTok{(Htwt}\SpecialCharTok{$}\NormalTok{wt}\SpecialCharTok{\textasciitilde{}}\NormalTok{Htwt}\SpecialCharTok{$}\NormalTok{ht))}
\end{Highlighting}
\end{Shaded}

\includegraphics{TD2RAA_files/figure-latex/unnamed-chunk-4-1.pdf}

\begin{enumerate}
\def\labelenumi{\alph{enumi}.}
\setcounter{enumi}{2}
\tightlist
\item
  Interpret the parameter estimates \(\hat{\beta_0}\) and
  \(\hat{\beta_0}\). Obtain the \(t\)-tests for the hypotheses that
  \(\beta_0=0\) and \(\beta_1=0\) and \(p\)-values using two-sided
  tests. What is your conclusion based on the \(p\)-values.
\end{enumerate}

\begin{Shaded}
\begin{Highlighting}[]
\FunctionTok{summary}\NormalTok{(lm\_Htwt)}
\end{Highlighting}
\end{Shaded}

\begin{verbatim}
## 
## Call:
## lm(formula = wt ~ ht, data = Htwt)
## 
## Residuals:
##     Min      1Q  Median      3Q     Max 
## -7.1166 -4.7744 -2.8412  0.5696 18.4581 
## 
## Coefficients:
##             Estimate Std. Error t value Pr(>|t|)
## (Intercept) -36.8759    64.4728  -0.572    0.583
## ht            0.5821     0.3892   1.496    0.173
## 
## Residual standard error: 8.456 on 8 degrees of freedom
## Multiple R-squared:  0.2185, Adjusted R-squared:  0.1208 
## F-statistic: 2.237 on 1 and 8 DF,  p-value: 0.1731
\end{verbatim}

\begin{enumerate}
\def\labelenumi{\alph{enumi}.}
\setcounter{enumi}{3}
\tightlist
\item
  Obtain \(R^2\) and adjusted \(R^2\). What can you say about the them?
\end{enumerate}

\begin{Shaded}
\begin{Highlighting}[]
\FunctionTok{summary}\NormalTok{(lm\_Htwt)}\SpecialCharTok{$}\NormalTok{r.squared}
\end{Highlighting}
\end{Shaded}

\begin{verbatim}
## [1] 0.2185191
\end{verbatim}

\begin{Shaded}
\begin{Highlighting}[]
\FunctionTok{summary}\NormalTok{(lm\_Htwt)}\SpecialCharTok{$}\NormalTok{adj.r.squared}
\end{Highlighting}
\end{Shaded}

\begin{verbatim}
## [1] 0.120834
\end{verbatim}

\begin{enumerate}
\def\labelenumi{\alph{enumi}.}
\setcounter{enumi}{4}
\tightlist
\item
  Check all the model assumptions for this simple linear regression.
\end{enumerate}

\begin{Shaded}
\begin{Highlighting}[]
\CommentTok{\#Linearity and independent of error}
\NormalTok{e}\OtherTok{\textless{}{-}}\NormalTok{lm\_Htwt}\SpecialCharTok{$}\NormalTok{residuals}
\NormalTok{y\_hat }\OtherTok{\textless{}{-}}\NormalTok{ lm\_Htwt}\SpecialCharTok{$}\NormalTok{fitted.values}

\FunctionTok{plot}\NormalTok{(y\_hat, e)}
\FunctionTok{abline}\NormalTok{(}\AttributeTok{h=}\DecValTok{0}\NormalTok{, }\AttributeTok{lty=}\DecValTok{2}\NormalTok{)}
\end{Highlighting}
\end{Shaded}

\includegraphics{TD2RAA_files/figure-latex/unnamed-chunk-7-1.pdf}

\begin{Shaded}
\begin{Highlighting}[]
\CommentTok{\#H\_0: Independence of Error}
\CommentTok{\#H\_A: The correlation exists}
\FunctionTok{durbinWatsonTest}\NormalTok{(lm\_Htwt)}
\end{Highlighting}
\end{Shaded}

\begin{verbatim}
##  lag Autocorrelation D-W Statistic p-value
##    1     -0.05772706      1.366839   0.406
##  Alternative hypothesis: rho != 0
\end{verbatim}

\begin{Shaded}
\begin{Highlighting}[]
\CommentTok{\# Normality}
\FunctionTok{hist}\NormalTok{(lm\_Htwt}\SpecialCharTok{$}\NormalTok{residuals)}
\end{Highlighting}
\end{Shaded}

\includegraphics{TD2RAA_files/figure-latex/unnamed-chunk-9-1.pdf}

\begin{Shaded}
\begin{Highlighting}[]
\FunctionTok{qqnorm}\NormalTok{(lm\_Htwt}\SpecialCharTok{$}\NormalTok{residuals)}
\FunctionTok{qqline}\NormalTok{(lm\_Htwt}\SpecialCharTok{$}\NormalTok{residuals)}
\end{Highlighting}
\end{Shaded}

\includegraphics{TD2RAA_files/figure-latex/unnamed-chunk-10-1.pdf}

\begin{Shaded}
\begin{Highlighting}[]
\CommentTok{\#H\_0: The data is normal}
\CommentTok{\#H\_A: The data is not normal }
\FunctionTok{shapiro.test}\NormalTok{(lm\_Htwt}\SpecialCharTok{$}\NormalTok{residuals)}
\end{Highlighting}
\end{Shaded}

\begin{verbatim}
## 
##  Shapiro-Wilk normality test
## 
## data:  lm_Htwt$residuals
## W = 0.78496, p-value = 0.009514
\end{verbatim}

\begin{Shaded}
\begin{Highlighting}[]
\CommentTok{\# Homoschedasticity }
\FunctionTok{plot}\NormalTok{(lm\_Htwt}\SpecialCharTok{$}\NormalTok{fitted.values, e)}
\end{Highlighting}
\end{Shaded}

\includegraphics{TD2RAA_files/figure-latex/unnamed-chunk-12-1.pdf}

\begin{Shaded}
\begin{Highlighting}[]
\FunctionTok{plot}\NormalTok{(lm\_Htwt}\SpecialCharTok{$}\NormalTok{fitted.values, e}\SpecialCharTok{\^{}}\DecValTok{2}\NormalTok{)}
\end{Highlighting}
\end{Shaded}

\includegraphics{TD2RAA_files/figure-latex/unnamed-chunk-13-1.pdf}

\begin{Shaded}
\begin{Highlighting}[]
\CommentTok{\#H\_0: Homoscedasticity holds}
\CommentTok{\#H\_A: The variance is not constant}
\FunctionTok{library}\NormalTok{(car)}
\FunctionTok{ncvTest}\NormalTok{(lm\_Htwt)}
\end{Highlighting}
\end{Shaded}

\begin{verbatim}
## Non-constant Variance Score Test 
## Variance formula: ~ fitted.values 
## Chisquare = 0.03970496, Df = 1, p = 0.84206
\end{verbatim}

\hypertarget{problem-2}{%
\section{Problem 2}\label{problem-2}}

(Data file: \texttt{UBSprices}) The international bank UBS regularly
produces a report (UBS, 2009) on prices and earnings in major cities
throughout the world. Three of the measures they include are prices of
basic commodities, namely 1kg of rice, a 1kg loaf of bread, and the
price of a Big Mac hamburger at McDonalds. An interesting feature of the
prices they report is that prices are measured in the minutes of labor
required for a ``typical'' worker in that location to earn enough money
to purchase the commodity. Using minutes of labor corrects at least in
part for currency fluctuations, prevailing wage rates, and local prices.
The data file includes measurements for rice, bread, and Big Mac prices
from the 2003 and the 2009 reports. The year 2003 was before the major
recession hit much of the world around 2006, and the year 2009 may
reflect changes in prices due to the recession. The figure below is the
plot of y = \texttt{rice2009} versus x = \texttt{rice2003}, the price of
rice in 2009 and 2003, respectively, with the cities corresponding to a
few of the points marked.

\begin{Shaded}
\begin{Highlighting}[]
\FunctionTok{library}\NormalTok{(alr4)}
\FunctionTok{data}\NormalTok{(}\StringTok{"UBSprices"}\NormalTok{)}
\end{Highlighting}
\end{Shaded}

\begin{Shaded}
\begin{Highlighting}[]
\FunctionTok{par}\NormalTok{(}\AttributeTok{mfrow=}\FunctionTok{c}\NormalTok{(}\DecValTok{1}\NormalTok{,}\DecValTok{2}\NormalTok{))}
\FunctionTok{plot}\NormalTok{(}\AttributeTok{x=}\NormalTok{UBSprices}\SpecialCharTok{$}\NormalTok{rice2003, }\AttributeTok{y=}\NormalTok{UBSprices}\SpecialCharTok{$}\NormalTok{rice2009,}
     \AttributeTok{xlab=}\StringTok{"2003 Rice price"}\NormalTok{, }
     \AttributeTok{ylab=}\StringTok{"2009 Rice price"}\NormalTok{)}
\CommentTok{\#identify(x=UBSprices$rice2003, y=UBSprices$rice2009, }
\CommentTok{\#         labels=row.names(UBSprices), n=5)}
\FunctionTok{abline}\NormalTok{(}\FunctionTok{lm}\NormalTok{(rice2009}\SpecialCharTok{\textasciitilde{}}\NormalTok{rice2003, }\AttributeTok{data=}\NormalTok{UBSprices), }\AttributeTok{lty=}\DecValTok{2}\NormalTok{)}
\FunctionTok{abline}\NormalTok{(}\AttributeTok{a=}\DecValTok{0}\NormalTok{, }\AttributeTok{b=}\DecValTok{1}\NormalTok{, }\AttributeTok{lty=}\DecValTok{1}\NormalTok{)}
\FunctionTok{legend}\NormalTok{(}\StringTok{"bottomright"}\NormalTok{, }\AttributeTok{legend=}\FunctionTok{c}\NormalTok{(}\StringTok{"ols"}\NormalTok{, }\StringTok{"y=x"}\NormalTok{), }\AttributeTok{lty=}\DecValTok{2}\SpecialCharTok{:}\DecValTok{1}\NormalTok{, }\AttributeTok{cex=}\FloatTok{0.6}\NormalTok{)}

\FunctionTok{plot}\NormalTok{(}\AttributeTok{x=}\NormalTok{UBSprices}\SpecialCharTok{$}\NormalTok{rice2003, }\AttributeTok{y=}\NormalTok{UBSprices}\SpecialCharTok{$}\NormalTok{rice2009,}
     \AttributeTok{xlab=}\StringTok{"2003 Rice price"}\NormalTok{, }
     \AttributeTok{ylab=}\StringTok{"2009 Rice price"}\NormalTok{)}
\FunctionTok{text}\NormalTok{(}\AttributeTok{x=}\NormalTok{UBSprices}\SpecialCharTok{$}\NormalTok{rice2003, }\AttributeTok{y=}\NormalTok{UBSprices}\SpecialCharTok{$}\NormalTok{rice2009,}
     \AttributeTok{labels=}\FunctionTok{row.names}\NormalTok{(UBSprices), }\AttributeTok{cex=}\FloatTok{0.6}\NormalTok{, }\AttributeTok{font=}\DecValTok{2}\NormalTok{)}
\FunctionTok{abline}\NormalTok{(}\FunctionTok{lm}\NormalTok{(rice2009}\SpecialCharTok{\textasciitilde{}}\NormalTok{rice2003, }\AttributeTok{data=}\NormalTok{UBSprices), }\AttributeTok{lty=}\DecValTok{2}\NormalTok{)}
\FunctionTok{abline}\NormalTok{(}\AttributeTok{a=}\DecValTok{0}\NormalTok{, }\AttributeTok{b=}\DecValTok{1}\NormalTok{, }\AttributeTok{lty=}\DecValTok{1}\NormalTok{)}
\FunctionTok{legend}\NormalTok{(}\StringTok{"bottomright"}\NormalTok{, }\AttributeTok{legend=}\FunctionTok{c}\NormalTok{(}\StringTok{"ols"}\NormalTok{, }\StringTok{"y=x"}\NormalTok{), }\AttributeTok{lty=}\DecValTok{2}\SpecialCharTok{:}\DecValTok{1}\NormalTok{, }\AttributeTok{cex=}\FloatTok{0.6}\NormalTok{)}
\end{Highlighting}
\end{Shaded}

\begin{center}\includegraphics{TD2RAA_files/figure-latex/unnamed-chunk-16-1} \end{center}

\begin{Shaded}
\begin{Highlighting}[]
\FunctionTok{par}\NormalTok{(}\AttributeTok{mfrow=}\FunctionTok{c}\NormalTok{(}\DecValTok{1}\NormalTok{,}\DecValTok{1}\NormalTok{))}
\end{Highlighting}
\end{Shaded}

\begin{enumerate}
\def\labelenumi{\alph{enumi}.}
\item
  The line with equation \(y = x\) is shown on this plot as the solid
  line. What is the key difference between points above this line and
  points below the line?
\item
  Which city had the largest increase in rice price? Which had the
  largest decrease in rice price?
\item
  The ols line \(\hat{y}=\hat{\beta}_0+\hat{\beta}_1x\) is shown on the
  figure as a dashed line, and evidently \(\hat{\beta}_1 <1\). Does this
  suggest that prices are lower in 2009 than in 2003? Explain your
  answer.
\item
  Give two reasons why fitting simple linear regression to the figure in
  this problem is not likely to be appropriate.
\end{enumerate}

\hypertarget{problem-3}{%
\section{Problem 3}\label{problem-3}}

(Data file: \texttt{UBSprices}) This is a continuation of Problem 2. An
alternative representation of the data used in the last problem is to
use log scales, as in the following figure:

\begin{Shaded}
\begin{Highlighting}[]
\FunctionTok{plot}\NormalTok{(}\AttributeTok{x=}\FunctionTok{log}\NormalTok{(UBSprices}\SpecialCharTok{$}\NormalTok{rice2003), }\AttributeTok{y=}\FunctionTok{log}\NormalTok{(UBSprices}\SpecialCharTok{$}\NormalTok{rice2009),}
     \AttributeTok{xlab=}\StringTok{"log(2003 Rice price)"}\NormalTok{, }
     \AttributeTok{ylab=}\StringTok{"log(2009 Rice price)"}\NormalTok{)}

\FunctionTok{abline}\NormalTok{(}\FunctionTok{lm}\NormalTok{(}\FunctionTok{log}\NormalTok{(rice2009)}\SpecialCharTok{\textasciitilde{}}\FunctionTok{log}\NormalTok{(rice2003), }\AttributeTok{data=}\NormalTok{UBSprices), }\AttributeTok{lty=}\DecValTok{2}\NormalTok{)}
\FunctionTok{abline}\NormalTok{(}\AttributeTok{a=}\DecValTok{0}\NormalTok{, }\AttributeTok{b=}\DecValTok{1}\NormalTok{, }\AttributeTok{lty=}\DecValTok{1}\NormalTok{)}
\FunctionTok{legend}\NormalTok{(}\StringTok{"bottomright"}\NormalTok{, }\AttributeTok{legend=}\FunctionTok{c}\NormalTok{(}\StringTok{"ols"}\NormalTok{, }\StringTok{"y=x"}\NormalTok{), }\AttributeTok{lty=}\DecValTok{2}\SpecialCharTok{:}\DecValTok{1}\NormalTok{, }\AttributeTok{cex=}\FloatTok{0.6}\NormalTok{)}
\end{Highlighting}
\end{Shaded}

\begin{center}\includegraphics{TD2RAA_files/figure-latex/unnamed-chunk-17-1} \end{center}

\begin{enumerate}
\def\labelenumi{\alph{enumi}.}
\item
  Explain why this graph and the graph in Problem 2 suggests that using
  log-scale is preferable if fitting simple linear regression is
  desired.
\item
  Suppose we start with a proposed model
\end{enumerate}

\[
E(y|x) = \gamma_0 x^{\beta_1}
\]

This is a common model in many areas of study. Examples include
allometry (Gould, 1966), where x could represent the size of one body
characteristic such as total weight and y represents some other body
characteristic, such as brain weight, psychophysics (Stevens, 1966), in
which x is a physical stimulus and y is a psychological response to it,
or in economics, where x could represent inputs and y outputs, where
this relationship is often called a Cobb-Douglas production function
(Greene, 2003).

If we take the logs of both sides of the last equation, we get

\[
\log(E(y|x)) = \log(\gamma_0) + \beta_1 \log(x)
\]

If we approximate \(\log(E(y|x)) \approx E(\log(y)|x)\), and write
\(\beta_0 = \log(\gamma_0)\), to the extent that the logarithm of the
expectation equals the expectation of the logarithm, we have

\[
E(\log(y)|x) = \beta_0 + \beta_1 \log(x)
\]

Give an interpretation of \(\beta_0\) and \(\beta_1\) in this setting,
assuming \(\beta_1>0\).

\hypertarget{problem-4}{%
\section{Problem 4}\label{problem-4}}

(Data file: \texttt{UBSprices}) This problem continues with the data
file \texttt{UBSprices} described in Problem 2.

\begin{enumerate}
\def\labelenumi{\alph{enumi}.}
\tightlist
\item
  Draw the plot of \texttt{y=bigmac2009} versus \texttt{x=bigmac2003},
  the price of a Big Mac hamburger in 2009 and 2003. On this plot draw
  (1) the ols fitted line; (2) the line \(y = x\). Identify the most
  unusual cases and describe why they are unusual.
\end{enumerate}

\begin{Shaded}
\begin{Highlighting}[]
\FunctionTok{plot}\NormalTok{(UBSprices}\SpecialCharTok{$}\NormalTok{bigmac2009}\SpecialCharTok{\textasciitilde{}}\NormalTok{UBSprices}\SpecialCharTok{$}\NormalTok{bigmac2003)}
\FunctionTok{abline}\NormalTok{(}\FunctionTok{lm}\NormalTok{(UBSprices}\SpecialCharTok{$}\NormalTok{bigmac2009}\SpecialCharTok{\textasciitilde{}}\NormalTok{UBSprices}\SpecialCharTok{$}\NormalTok{bigmac2003))}
\FunctionTok{abline}\NormalTok{(}\DecValTok{0}\NormalTok{,}\DecValTok{1}\NormalTok{, }\AttributeTok{col=}\StringTok{"red"}\NormalTok{, }\AttributeTok{lty=}\DecValTok{2}\NormalTok{)}
\FunctionTok{legend}\NormalTok{(}\StringTok{"bottomright"}\NormalTok{, }\AttributeTok{inset=}\FloatTok{0.02}\NormalTok{, }\AttributeTok{legend=}\FunctionTok{c}\NormalTok{(}\StringTok{"OLS"}\NormalTok{, }\StringTok{"y=x"}\NormalTok{),}\AttributeTok{col=}\FunctionTok{c}\NormalTok{(}\StringTok{"black"}\NormalTok{, }\StringTok{"red"}\NormalTok{), }\AttributeTok{lty=}\DecValTok{1}\SpecialCharTok{:}\DecValTok{2}\NormalTok{, }\AttributeTok{cex=}\FloatTok{0.8}\NormalTok{)}
\end{Highlighting}
\end{Shaded}

\includegraphics{TD2RAA_files/figure-latex/unnamed-chunk-18-1.pdf}

\begin{Shaded}
\begin{Highlighting}[]
\FunctionTok{plot}\NormalTok{(UBSprices}\SpecialCharTok{$}\NormalTok{bigmac2009}\SpecialCharTok{\textasciitilde{}}\NormalTok{UBSprices}\SpecialCharTok{$}\NormalTok{bigmac2003)}
\end{Highlighting}
\end{Shaded}

\includegraphics{TD2RAA_files/figure-latex/unnamed-chunk-19-1.pdf}

\begin{Shaded}
\begin{Highlighting}[]
\FunctionTok{plot}\NormalTok{(UBSprices}\SpecialCharTok{$}\NormalTok{bigmac2009}\SpecialCharTok{\textasciitilde{}}\NormalTok{UBSprices}\SpecialCharTok{$}\NormalTok{bigmac2003)}
\FunctionTok{abline}\NormalTok{(}\FunctionTok{lm}\NormalTok{(UBSprices}\SpecialCharTok{$}\NormalTok{bigmac2009}\SpecialCharTok{\textasciitilde{}}\NormalTok{UBSprices}\SpecialCharTok{$}\NormalTok{bigmac2003))}
\FunctionTok{abline}\NormalTok{(}\DecValTok{0}\NormalTok{,}\DecValTok{1}\NormalTok{, }\AttributeTok{col=}\StringTok{"red"}\NormalTok{, }\AttributeTok{lty=}\DecValTok{2}\NormalTok{)}
\FunctionTok{legend}\NormalTok{(}\StringTok{"bottomright"}\NormalTok{, }\AttributeTok{inset=}\FloatTok{0.02}\NormalTok{, }\AttributeTok{legend=}\FunctionTok{c}\NormalTok{(}\StringTok{"OLS"}\NormalTok{, }\StringTok{"y=x"}\NormalTok{),}
       \AttributeTok{col=}\FunctionTok{c}\NormalTok{(}\StringTok{"black"}\NormalTok{, }\StringTok{"red"}\NormalTok{), }\AttributeTok{lty=}\DecValTok{1}\SpecialCharTok{:}\DecValTok{2}\NormalTok{, }\AttributeTok{cex=}\FloatTok{0.8}\NormalTok{)}
\end{Highlighting}
\end{Shaded}

\includegraphics{TD2RAA_files/figure-latex/unnamed-chunk-20-1.pdf}

\begin{Shaded}
\begin{Highlighting}[]
\FunctionTok{plot}\NormalTok{(UBSprices}\SpecialCharTok{$}\NormalTok{bigmac2009}\SpecialCharTok{\textasciitilde{}}\NormalTok{UBSprices}\SpecialCharTok{$}\NormalTok{bigmac2003)}
\CommentTok{\#identify(x=UBSprices$bigmac2003, y=UBSprices$bigmac2009, }
\CommentTok{\#         labels=row.names(UBSprices), n=3)}
\FunctionTok{text}\NormalTok{(}\AttributeTok{x=}\NormalTok{UBSprices}\SpecialCharTok{$}\NormalTok{bigmac2003, }\AttributeTok{y=}\NormalTok{UBSprices}\SpecialCharTok{$}\NormalTok{bigmac2009,}
     \AttributeTok{labels=}\FunctionTok{row.names}\NormalTok{(UBSprices), }\AttributeTok{cex=}\FloatTok{0.6}\NormalTok{, }\AttributeTok{font=}\DecValTok{2}\NormalTok{)}
\FunctionTok{abline}\NormalTok{(}\FunctionTok{lm}\NormalTok{(UBSprices}\SpecialCharTok{$}\NormalTok{bigmac2009}\SpecialCharTok{\textasciitilde{}}\NormalTok{UBSprices}\SpecialCharTok{$}\NormalTok{bigmac2003))}
\FunctionTok{abline}\NormalTok{(}\DecValTok{0}\NormalTok{,}\DecValTok{1}\NormalTok{, }\AttributeTok{col=}\StringTok{"red"}\NormalTok{, }\AttributeTok{lty=}\DecValTok{2}\NormalTok{)}
\FunctionTok{legend}\NormalTok{(}\StringTok{"bottomright"}\NormalTok{, }\AttributeTok{inset=}\FloatTok{0.02}\NormalTok{, }\AttributeTok{legend=}\FunctionTok{c}\NormalTok{(}\StringTok{"OLS"}\NormalTok{, }\StringTok{"y=x"}\NormalTok{),}
       \AttributeTok{col=}\FunctionTok{c}\NormalTok{(}\StringTok{"black"}\NormalTok{, }\StringTok{"red"}\NormalTok{), }\AttributeTok{lty=}\DecValTok{1}\SpecialCharTok{:}\DecValTok{2}\NormalTok{, }\AttributeTok{cex=}\FloatTok{0.8}\NormalTok{)}
\end{Highlighting}
\end{Shaded}

\includegraphics{TD2RAA_files/figure-latex/unnamed-chunk-20-2.pdf}

\begin{enumerate}
\def\labelenumi{\alph{enumi}.}
\setcounter{enumi}{1}
\tightlist
\item
  Give two reasons why fitting simple linear regression to the figure in
  this problem is not likely to be appropriate.
\end{enumerate}

\begin{Shaded}
\begin{Highlighting}[]
\FunctionTok{hist}\NormalTok{(UBSprices}\SpecialCharTok{$}\NormalTok{bigmac2009)}
\end{Highlighting}
\end{Shaded}

\includegraphics{TD2RAA_files/figure-latex/unnamed-chunk-21-1.pdf}

\begin{Shaded}
\begin{Highlighting}[]
\FunctionTok{hist}\NormalTok{(UBSprices}\SpecialCharTok{$}\NormalTok{bigmac2003)}
\end{Highlighting}
\end{Shaded}

\includegraphics{TD2RAA_files/figure-latex/unnamed-chunk-22-1.pdf}

\begin{enumerate}
\def\labelenumi{\alph{enumi}.}
\setcounter{enumi}{2}
\tightlist
\item
  Plot log(\texttt{bigmac2009}) versus log(\texttt{bigmac2003}) and
  explain why this graph is more sensibly summarized with a linear
  regression.
\end{enumerate}

\begin{Shaded}
\begin{Highlighting}[]
\FunctionTok{plot}\NormalTok{(}\FunctionTok{log}\NormalTok{(UBSprices}\SpecialCharTok{$}\NormalTok{bigmac2009)}\SpecialCharTok{\textasciitilde{}}\FunctionTok{log}\NormalTok{(UBSprices}\SpecialCharTok{$}\NormalTok{bigmac2003))}
\FunctionTok{abline}\NormalTok{(}\FunctionTok{lm}\NormalTok{(}\FunctionTok{log}\NormalTok{(UBSprices}\SpecialCharTok{$}\NormalTok{bigmac2009)}\SpecialCharTok{\textasciitilde{}}\FunctionTok{log}\NormalTok{(UBSprices}\SpecialCharTok{$}\NormalTok{bigmac2003)))}
\end{Highlighting}
\end{Shaded}

\includegraphics{TD2RAA_files/figure-latex/unnamed-chunk-23-1.pdf}

\hypertarget{problem-5}{%
\section{Problem 5}\label{problem-5}}

\textbf{Ft. Collins temperature data} (Data file:
\texttt{ftcollinstemp}) The data file gives the mean temperature in the
\texttt{fall} of each year, defined as September 1 to November 30, and
the mean temperature in the following \texttt{winter}, defined as
December 1 to the end of February in the following calendar year, in
degrees Fahrenheit, for Ft. Collins, CO (Colorado Climate Center, 2012).
These data cover the time period from 1900 to 2010. The question of
interest is: Does the average \texttt{fall} temperature predict the
average \texttt{winter} temperature?

\begin{Shaded}
\begin{Highlighting}[]
\FunctionTok{library}\NormalTok{(alr4)}
\FunctionTok{data}\NormalTok{(}\StringTok{"ftcollinstemp"}\NormalTok{)}
\FunctionTok{head}\NormalTok{(ftcollinstemp)}
\end{Highlighting}
\end{Shaded}

\begin{verbatim}
##   year fall winter
## 1 1900 50.2   28.5
## 2 1901 48.1   28.2
## 3 1902 48.1   24.7
## 4 1903 49.9   31.4
## 5 1904 47.8   26.7
## 6 1905 46.9   29.8
\end{verbatim}

\begin{enumerate}
\def\labelenumi{\alph{enumi}.}
\tightlist
\item
  Draw a scatterplot of the response versus the predictor, and describe
  any pattern you might see in the plot.
\end{enumerate}

\begin{Shaded}
\begin{Highlighting}[]
\FunctionTok{plot}\NormalTok{(ftcollinstemp}\SpecialCharTok{$}\NormalTok{winter}\SpecialCharTok{\textasciitilde{}}\NormalTok{ftcollinstemp}\SpecialCharTok{$}\NormalTok{fall)}
\FunctionTok{abline}\NormalTok{(}\FunctionTok{lm}\NormalTok{(ftcollinstemp}\SpecialCharTok{$}\NormalTok{winter}\SpecialCharTok{\textasciitilde{}}\NormalTok{ftcollinstemp}\SpecialCharTok{$}\NormalTok{fall))}
\end{Highlighting}
\end{Shaded}

\includegraphics{TD2RAA_files/figure-latex/unnamed-chunk-25-1.pdf}

\begin{Shaded}
\begin{Highlighting}[]
\FunctionTok{library}\NormalTok{(dplyr)}
\end{Highlighting}
\end{Shaded}

\begin{verbatim}
## 
## Attaching package: 'dplyr'
\end{verbatim}

\begin{verbatim}
## The following object is masked from 'package:car':
## 
##     recode
\end{verbatim}

\begin{verbatim}
## The following objects are masked from 'package:stats':
## 
##     filter, lag
\end{verbatim}

\begin{verbatim}
## The following objects are masked from 'package:base':
## 
##     intersect, setdiff, setequal, union
\end{verbatim}

\begin{Shaded}
\begin{Highlighting}[]
\FunctionTok{library}\NormalTok{(ggplot2)}
\NormalTok{c.temp}\OtherTok{\textless{}{-}}\NormalTok{ftcollinstemp }\SpecialCharTok{\%\textgreater{}\%}
  \FunctionTok{group\_by}\NormalTok{(fall) }\SpecialCharTok{\%\textgreater{}\%} 
  \FunctionTok{summarise\_at}\NormalTok{(}\FunctionTok{vars}\NormalTok{(winter), }\FunctionTok{list}\NormalTok{(}\AttributeTok{mean =}\NormalTok{ mean, }\AttributeTok{sd=}\NormalTok{sd))}

\FunctionTok{ggplot}\NormalTok{(c.temp, }\FunctionTok{aes}\NormalTok{(fall,mean)) }\SpecialCharTok{+} 
  \FunctionTok{geom\_point}\NormalTok{() }\SpecialCharTok{+}
  \FunctionTok{geom\_errorbar}\NormalTok{(}\FunctionTok{aes}\NormalTok{(}\AttributeTok{ymin=}\NormalTok{mean}\SpecialCharTok{{-}}\NormalTok{sd, }\AttributeTok{ymax=}\NormalTok{mean}\SpecialCharTok{+}\NormalTok{sd), }\AttributeTok{width=}\FloatTok{0.2}\NormalTok{, }
                \AttributeTok{position=}\FunctionTok{position\_dodge}\NormalTok{(}\FloatTok{0.05}\NormalTok{)) }\SpecialCharTok{+}
  \FunctionTok{labs}\NormalTok{(}\AttributeTok{x=}\StringTok{"fall"}\NormalTok{, }\AttributeTok{y=}\StringTok{"average winter"}\NormalTok{) }\SpecialCharTok{+} 
  \FunctionTok{theme\_bw}\NormalTok{()}
\end{Highlighting}
\end{Shaded}

\begin{verbatim}
## Warning: `position_dodge()` requires non-overlapping x intervals
\end{verbatim}

\includegraphics{TD2RAA_files/figure-latex/unnamed-chunk-25-2.pdf}

\begin{enumerate}
\def\labelenumi{\alph{enumi}.}
\setcounter{enumi}{1}
\tightlist
\item
  Use statistical software to fit the regression of the response on the
  predictor. Add the fitted line to your graph. Test the slope to be 0
  against a two-sided alternative, and summarize your results.
\end{enumerate}

\begin{Shaded}
\begin{Highlighting}[]
\NormalTok{lm\_temp }\OtherTok{\textless{}{-}} \FunctionTok{lm}\NormalTok{(winter}\SpecialCharTok{\textasciitilde{}}\NormalTok{fall, }\AttributeTok{data=}\NormalTok{ftcollinstemp)}
\FunctionTok{summary}\NormalTok{(lm\_temp)}
\end{Highlighting}
\end{Shaded}

\begin{verbatim}
## 
## Call:
## lm(formula = winter ~ fall, data = ftcollinstemp)
## 
## Residuals:
##     Min      1Q  Median      3Q     Max 
## -7.8186 -1.7837 -0.0873  2.1300  7.5896 
## 
## Coefficients:
##             Estimate Std. Error t value Pr(>|t|)  
## (Intercept)  13.7843     7.5549   1.825   0.0708 .
## fall          0.3132     0.1528   2.049   0.0428 *
## ---
## Signif. codes:  0 '***' 0.001 '**' 0.01 '*' 0.05 '.' 0.1 ' ' 1
## 
## Residual standard error: 3.179 on 109 degrees of freedom
## Multiple R-squared:  0.0371, Adjusted R-squared:  0.02826 
## F-statistic:   4.2 on 1 and 109 DF,  p-value: 0.04284
\end{verbatim}

\begin{enumerate}
\def\labelenumi{\alph{enumi}.}
\setcounter{enumi}{2}
\tightlist
\item
  Compute or obtain from your computer output the value of the
  variability in \texttt{winter} explained by \texttt{fall} and explain
  what this means.
\end{enumerate}

\begin{Shaded}
\begin{Highlighting}[]
\FunctionTok{summary}\NormalTok{(lm\_temp)}\SpecialCharTok{$}\NormalTok{r.squared}
\end{Highlighting}
\end{Shaded}

\begin{verbatim}
## [1] 0.03709854
\end{verbatim}

\begin{enumerate}
\def\labelenumi{\alph{enumi}.}
\setcounter{enumi}{3}
\tightlist
\item
  Divide the data into 2 time periods, an early period from 1900 to
  1989, and a late period from 1990 to 2010. You can do this using the
  variable \texttt{year} in the data file. Are the results different in
  the two time periods?
\end{enumerate}

\begin{Shaded}
\begin{Highlighting}[]
\NormalTok{temp1989 }\OtherTok{\textless{}{-}} \FunctionTok{filter}\NormalTok{(ftcollinstemp, year}\SpecialCharTok{\textless{}=}\DecValTok{1989}\NormalTok{)}
\NormalTok{temp2010 }\OtherTok{\textless{}{-}} \FunctionTok{filter}\NormalTok{(ftcollinstemp, year}\SpecialCharTok{\textgreater{}=}\DecValTok{1990}\NormalTok{)}
\FunctionTok{nrow}\NormalTok{(temp1989)}
\end{Highlighting}
\end{Shaded}

\begin{verbatim}
## [1] 90
\end{verbatim}

\begin{Shaded}
\begin{Highlighting}[]
\FunctionTok{nrow}\NormalTok{(temp2010)}
\end{Highlighting}
\end{Shaded}

\begin{verbatim}
## [1] 21
\end{verbatim}

\begin{Shaded}
\begin{Highlighting}[]
\FunctionTok{nrow}\NormalTok{(ftcollinstemp)}
\end{Highlighting}
\end{Shaded}

\begin{verbatim}
## [1] 111
\end{verbatim}

\begin{Shaded}
\begin{Highlighting}[]
\FunctionTok{plot}\NormalTok{(winter}\SpecialCharTok{\textasciitilde{}}\NormalTok{fall, }\AttributeTok{data=}\NormalTok{temp1989)}
\end{Highlighting}
\end{Shaded}

\includegraphics{TD2RAA_files/figure-latex/unnamed-chunk-29-1.pdf}

\begin{Shaded}
\begin{Highlighting}[]
\NormalTok{lm\_1989}\OtherTok{\textless{}{-}}\FunctionTok{lm}\NormalTok{(winter}\SpecialCharTok{\textasciitilde{}}\NormalTok{fall, }\AttributeTok{data=}\NormalTok{temp1989)}
\FunctionTok{summary}\NormalTok{(lm\_1989)}
\end{Highlighting}
\end{Shaded}

\begin{verbatim}
## 
## Call:
## lm(formula = winter ~ fall, data = temp1989)
## 
## Residuals:
##     Min      1Q  Median      3Q     Max 
## -6.8976 -1.6349  0.0118  2.0079  7.3387 
## 
## Coefficients:
##             Estimate Std. Error t value Pr(>|t|)   
## (Intercept)  22.7079     8.2600   2.749  0.00725 **
## fall          0.1209     0.1681   0.719  0.47397   
## ---
## Signif. codes:  0 '***' 0.001 '**' 0.01 '*' 0.05 '.' 0.1 ' ' 1
## 
## Residual standard error: 3.057 on 88 degrees of freedom
## Multiple R-squared:  0.005842,   Adjusted R-squared:  -0.005455 
## F-statistic: 0.5171 on 1 and 88 DF,  p-value: 0.474
\end{verbatim}

\begin{Shaded}
\begin{Highlighting}[]
\FunctionTok{plot}\NormalTok{(winter}\SpecialCharTok{\textasciitilde{}}\NormalTok{fall, }\AttributeTok{data=}\NormalTok{temp2010)}
\end{Highlighting}
\end{Shaded}

\includegraphics{TD2RAA_files/figure-latex/unnamed-chunk-30-1.pdf}

\begin{Shaded}
\begin{Highlighting}[]
\NormalTok{lm\_2010}\OtherTok{\textless{}{-}}\FunctionTok{lm}\NormalTok{(winter}\SpecialCharTok{\textasciitilde{}}\NormalTok{fall, }\AttributeTok{data=}\NormalTok{temp2010)}
\FunctionTok{summary}\NormalTok{(lm\_2010)}
\end{Highlighting}
\end{Shaded}

\begin{verbatim}
## 
## Call:
## lm(formula = winter ~ fall, data = temp2010)
## 
## Residuals:
##     Min      1Q  Median      3Q     Max 
## -5.4174 -1.7097  0.3768  1.8988  4.9602 
## 
## Coefficients:
##             Estimate Std. Error t value Pr(>|t|)
## (Intercept)  24.8260    17.7973   1.395    0.179
## fall          0.1390     0.3509   0.396    0.696
## 
## Residual standard error: 2.699 on 19 degrees of freedom
## Multiple R-squared:  0.00819,    Adjusted R-squared:  -0.04401 
## F-statistic: 0.1569 on 1 and 19 DF,  p-value: 0.6965
\end{verbatim}

\end{document}
